\documentclass[10pt]{article}
\usepackage{fullpage}
\usepackage{amsmath}
\usepackage{amssymb}
\usepackage[dvips]{graphicx}


%opening
\title{Correlated Luminosity Function}
\author{Alan}
\date{23 Oct 2014}


\begin{document}
\maketitle

I now have code to select all neighboring galaxies within a given radius on the sky from the SDSS CasJobs database. The data are from DR10.
I have downloaded this data for a small subset of galaxies (about 10,000) located at $0.500\leq z \leq 0.505$. For each galaxy, I search out to 
3 arc-minutes. For 10,000 galaxies, this returns about 2 million objects, or about 200 neighbors per galaxy. I collect the PetroMag, CModelMag, ModelMag, 
DevMag, ExpMag, and Fracdev for each galaxy in the g, r, and i bands.

Using the separation and the redshift, I calculate the projected physical separation in kpc assuming the neighbor is at the same redshift as the target 
galaxy. For the plots below, I separate into 4 bins: 
\begin{enumerate}
 \item $100\leq distance \leq 300$ kpc
 \item $300\leq distance \leq 500$ kpc
 \item $500\leq distance \leq 700$ kpc
 \item $700\leq distance \leq 900$ kpc
\end{enumerate}
I plot the distribution of magnitudes for the g (in green), r (in red), and i (in black) bands for the CMASS galaxies (the darker lines) and for the blank
sky patches (the lighter lines). The difference is easiest to see in the i-band, but it appears that the distribution is shifted toward brighter magnitudes
for the CMASS galaxies relative to the blank sky (the darker black line is above the light black line to the left of the maximum and below to the right 
of the maximum). There is also a second peak around 24-25 magnitude. This is near the detection limit. I am not sure what these objects are yet.


\begin{figure}
 \includegraphics[width=0.5\textwidth]{corr_lum_all_100_appmag_dist.eps}
 \includegraphics[width=0.5\textwidth]{corr_lum_all_300_appmag_dist.eps}
 \includegraphics[width=0.5\textwidth]{corr_lum_all_500_appmag_dist.eps}
 \includegraphics[width=0.5\textwidth]{corr_lum_all_700_appmag_dist.eps}
\caption{Appmag distribution using all primary objects and random sky locations 
(ie all stars and galaxies, blank sky positions are not centered on an object).}
\end{figure}

\begin{figure}
 \includegraphics[width=0.5\textwidth]{corr_lum_all_100_absmag.eps}
 \includegraphics[width=0.5\textwidth]{corr_lum_all_300_absmag.eps}
 \includegraphics[width=0.5\textwidth]{corr_lum_all_500_absmag.eps}
 \includegraphics[width=0.5\textwidth]{corr_lum_all_700_absmag.eps}
\caption{Absmag distribution using all primary objects and random sky locations 
(ie all stars and galaxies, blank sky positions are not centered on an object).}
\end{figure}

\begin{figure}
 \includegraphics[width=0.5\textwidth]{corr_lum_all_100_lum_func.eps}
 \includegraphics[width=0.5\textwidth]{corr_lum_all_300_lum_func.eps}
 \includegraphics[width=0.5\textwidth]{corr_lum_all_500_lum_func.eps}
 \includegraphics[width=0.5\textwidth]{corr_lum_all_700_lum_func.eps}
\caption{Luminosity function using all primary objects and random sky locations 
(ie all stars and galaxies, blank sky positions are not centered on an object).}
\end{figure}

\begin{figure}
 \includegraphics[width=0.5\textwidth]{corr_lum_all_galaxy_100_appmag_dist.eps}
 \includegraphics[width=0.5\textwidth]{corr_lum_all_galaxy_300_appmag_dist.eps}
 \includegraphics[width=0.5\textwidth]{corr_lum_all_galaxy_500_appmag_dist.eps}
 \includegraphics[width=0.5\textwidth]{corr_lum_all_galaxy_700_appmag_dist.eps}
\caption{Appmag distribution using only primary Galaxies and random sky locations 
(ie only galaxies, blank sky positions are not centered on an object).}
\end{figure}

\begin{figure}
 \includegraphics[width=0.5\textwidth]{corr_lum_all_galaxy_100_absmag.eps}
 \includegraphics[width=0.5\textwidth]{corr_lum_all_galaxy_300_absmag.eps}
 \includegraphics[width=0.5\textwidth]{corr_lum_all_galaxy_500_absmag.eps}
 \includegraphics[width=0.5\textwidth]{corr_lum_all_galaxy_700_absmag.eps}
\caption{Absmag distribution using only primary Galaxies and random sky locations 
(ie only galaxies, blank sky positions are not centered on an object).}
\end{figure}

\begin{figure}
 \includegraphics[width=0.5\textwidth]{corr_lum_all_galaxy_100_lum_func.eps}
 \includegraphics[width=0.5\textwidth]{corr_lum_all_galaxy_300_lum_func.eps}
 \includegraphics[width=0.5\textwidth]{corr_lum_all_galaxy_500_lum_func.eps}
 \includegraphics[width=0.5\textwidth]{corr_lum_all_galaxy_700_lum_func.eps}
\caption{Luminosity function using only primary Galaxies and random sky locations 
(ie only galaxies, blank sky positions are not centered on an object).}
\end{figure}


\end{document}